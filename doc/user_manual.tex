\documentclass{article}

\usepackage{xcolor}

\newcommand{\eq}[1]{(\ref{#1})}

\newcommand{\code}[1]{\texttt{#1}}
\newcommand{\class}[1]{\colorbox{blue!30}{\code{#1}}}

\newcommand{\ns}{n_{\mathrm{s}}}
\newcommand{\ps}{\vec{p}_{\mathrm{s}}}
\newcommand{\xs}{\vec{x}_{\mathrm{s}}}
\newcommand{\xsk}{\vec{x}_{\mathrm{s},k}}

\newcommand{\dPhisdE}{\frac{\mathrm{d}\Phi_{\mathrm{s}}}{\mathrm{d}E}}


\begin{document}

\section{The Likelihood Formalism}

This section describes the mathematical likelihood formalism used in Skylab.
First it introduces the log-likelihood approach, second the likelihood ratio
test and the used test statistic and then describes the used optimizations.

\subsection{The Log-Likelihood Approach}

Skylab implements the two-component likelihood approach with a likelihood
function $\mathcal{L}(n_{\mathrm{s}},\vec{p}_{\mathrm{s}}~|D)$ of the form
\begin{equation}
 \mathcal{L}(\ns,\ps~|D) = \prod_{i=1}^{N}\left[ \frac{\ns}{N} \mathcal{S}_{i}(\vec{x_{\mathrm{s}}},\ps) + (1 - \frac{\ns}{N}) \mathcal{B}_{i} \right],
\label{eq:L}
\end{equation}
where $\ns$ is the number of signal events, hence, $(N-\ns)$ the number of
background events in the dataset $D$ of $N$ total events.
The set of signal model parameters is denoted as $\ps$. Signal model parameters
are for instance the spectral index $\gamma$ of the source.
$\mathcal{S}_i(\vec{x}_{\mathrm{s}},\ps)$ is the value of the signal PDF
assuming a signal source at location $\vec{x}_{\mathrm{s}}$ for the $i$th data
event, whereas $\mathcal{B}_i$ is the value of background PDF of the $i$th data
event.

The signal and background PDFs must incorperate the detector efficiency (yield)
$\mathcal{Y}_i$, which usually is dependent on the data event's sky location,
energy, and time.

For computational stability reasons the logarithm of the likelihood function of
equation \ref{eq:L} is used in Skylab:
\begin{equation}
 \log \mathcal{L}(\ns,\ps~|D) = \sum_{i=1}^{N} \log (...)
\end{equation}

\subsection{Likelihood Ratio Test and Test Statistic}

For estimating the significance of an observation, the likelihood ratio
$\Lambda$ with respect to a null hypothesis of no observation, i.e.
equation \ref{eq:L} at $\ns=0$ is tested:
\begin{equation}
 \log \Lambda(\ns,\ps) = \log \frac{L(\ns,\ps)}{L(\ns=0)} = \sum_{i=1}^{N} \log \left[ 1 + \frac{\ns}{N}\left( \frac{\mathcal{S}_i(\xs,\ps)}{\mathcal{B}_i} - 1 \right) \right]
\end{equation}
By defining
\begin{equation}
\mathcal{X}_i(\ps) \equiv \frac{1}{N}\left( \frac{\mathcal{S}_i(\xs,\ps)}{\mathcal{B}_i} - 1 \right),
\label{eq:Xi}
\end{equation}
this reads as:
\begin{equation}
 \log \Lambda(\ns,\ps) = \sum_{i=1}^{N} \log (1 + \ns\mathcal{X}_i(\ps))
 \label{eq:logLambda}
\end{equation}
This leads to the test statistic TS
\begin{equation}
 \mathrm{TS} = 2\mathrm{sgn}(\ns) \log \Lambda(\ns,\ps)
 \label{eq:TS}
\end{equation}
with separation of over- ($\ns > 0$) and under-fluctuations ($\ns < 0$).

\subsection{Multiple Datasets}

With Skylab a set of $J$ different data samples (datasets) $\mathrm{D}_j$ can be
analyzed at once. Each data sample has its own detector signal efficiency
$\mathcal{Y}_{\mathrm{s},j}$.

The composite likelihood function is the product of the individual dataset
likelihood functions:
\begin{equation}
 \log \Lambda = \sum_{j=1}^{J} \log \Lambda_j
 \label{eq:logLambdaComposite}
\end{equation}

The total number of signal events $\ns$ needs to get split-up into
$n_{\mathrm{s},j}$ for the individual data samples. The distribution of $\ns$
along the different data samples is based on the detector signal efficiency
$\mathcal{Y}_{\mathrm{s},j}$ of each sample:
\begin{equation}
 n_{s,j}(\ps~|\vec{x}_{\mathrm{s}}) = \ns \frac{\mathcal{Y}_{\mathrm{s},j}(\vec{x}_{\mathrm{s}},\ps)}{\sum_j \mathcal{Y}_{\mathrm{s},j}(\vec{x}_{\mathrm{s}},\ps)},
 \label{eq:nsjy}
\end{equation}
where the parameter $\vec{x}_{\mathrm{s}}$ denotes the location(s) of the source(s).
The parameter vector $\ps$ contains additional source hypothesis parameters,
for instance the spectral index $\gamma$.

By defining the sample weight factor
\begin{equation}
 f_j(\ps~|\vec{x}_{\mathrm{s}}) \equiv \frac{\mathcal{Y}_{\mathrm{s},j}(\vec{x}_{\mathrm{s}},\ps)}{\sum_j \mathcal{Y}_{\mathrm{s},j}(\vec{x}_{\mathrm{s}},\ps)}
\end{equation}
with the property
\begin{equation}
 \sum_{j=1}^{J} f_j = 1
\end{equation}
equation \ref{eq:nsjy} reads
\begin{equation}
 n_{s,j}(\ps~|\vec{x}_{\mathrm{s}}) = \ns f_{j}(\ps~|\vec{x}_{\mathrm{s}})
 \label{eq:ns-sample-weight-factor}
\end{equation}

The detector signal efficiency $\mathcal{Y}_{\mathrm{s},j}(\ps,\xs)$
depends on the signal model parameters $\ps$ and the source location(s) $\xs$
and can be calculated via the detector effective area and the source flux.

For a single point source the sample weight factor can be calculated via the effective area
$A_{\mathrm{eff},j}(E)|_{\xs}$ at the source location of each data sample, and the
differential flux $\dPhisdE$ of the source.
\begin{equation}
 f_{j}(\ps~|\xs) = \frac{\int_0^\infty \mathrm{d}E A_{\mathrm{eff},j}(E)|_{\xs} \dPhisdE(E,\ps)}
		        {\sum_{i=1}^{J} \int_0^\infty \mathrm{d}E A_{\mathrm{eff},i}(E)|_{\xs} \dPhisdE(E,\ps)}
 \label{eq:fj}
\end{equation}

Using the sample weight factor $f_{j}(\ps~|\xs)$ the likelihood ratio of
equation \eq{eq:logLambdaComposite} with equation \eq{eq:logLambda} can now
be written as
\begin{equation}
 \log \Lambda(\ns,\ps) = \sum_{j=1}^{J} \sum_{i=1}^{N} \log (1 + \ns f_{j}(\ps~|\xs)\mathcal{X}_i(\ps))
\end{equation}


For multiple point sources, i.e. a stacking of $K$ point sources with positions
$\vec{x}_{\mathrm{s},k}$, the sample weight factor of each single source needs
to be taking into account. Thus, $f_{j}$ can be written as the sum of the
products of the sample weight factor $f_{j}(\ps~|\vec{x}_{\mathrm{s},k})$ for
source $k$ and the relative strength $f_{k}(\ps~|\vec{x}_{\mathrm{s},k})$ of the
$k$th source in all data samples compared to all the other sources in all data
samples.
\begin{equation}
 f_{j}(\ps~|\xs) = \sum_{k=1}^{K} f_{j}(\ps~|\vec{x}_{\mathrm{s},k}) f_{k}(\ps~|\vec{x}_{\mathrm{s},k})
\end{equation}
The relative strength of source $k$ can be written as
\begin{equation}
 f_{k}(\ps~|\vec{x}_{\mathrm{s},k}) =
    \frac{\sum_{i=1}^{J} \int_0^\infty \mathrm{d}E A_{\mathrm{eff},i}(E)|_{\vec{x}_{\mathrm{s},k}} \dPhisdE(E,\ps)}
         {\sum_{\kappa=1}^{K} \sum_{i=1}^{J} \int_0^\infty \mathrm{d}E A_{\mathrm{eff},i}(E)|_{\vec{x}_{\mathrm{s},\kappa}} \dPhisdE(E,\ps) }
 \label{eq:fk}
\end{equation}
Combining equation \ref{eq:fj} with $\xs \equiv \vec{x}_{\mathrm{s},k}$ and
\ref{eq:fk} leads to the final expression for $f_{j}$ for multiple sources:
\begin{equation}
 f_{j}(\ps~|\xs) = \frac{\sum_{k=1}^{K} \int_0^\infty \mathrm{d}E A_{\mathrm{eff},j}(E)|_{\vec{x}_{\mathrm{s},k}} \dPhisdE(E,\ps) }
                        {\sum_{i=1}^{J} \sum_{k=1}^{K} \int_0^\infty \mathrm{d}E A_{\mathrm{eff},i}(E)|_{\vec{x}_{\mathrm{s},k}} \dPhisdE(E,\ps) }
 \label{eq:fjmultips}
\end{equation}
One should note that the numerator of equation \eq{eq:fjmultips} is one of the
summands of the sum in the denumerator, i.e. for $i=j$.

\subsection{Optimizations}

For point-source like signal hypothesis most of the events in the data sample
will be far away from the hypothesised source, hence, the value of the
signal PDF $\mathcal{S}_i$ will be zero or very close to zero. By selecting only
the signal-contributing $N'$ events from the sample, the likelihood ratio
$\log \Lambda$ reads
\begin{equation}
 \log \Lambda = \log \Lambda_{N'} + (N - N')\log(1 - \frac{\ns}{N})
\end{equation}

The used minimizer (L-BFG-S) operates most stable and fast when provided with
gradients of the likelihood function.

TODO: Derive the gradients of the LH function.

\section{Detector Signal Efficiency}

The detector signal efficiency $\mathcal{Y}_{\mathrm{s},j}(\vec{x}_{\mathrm{s},k},\ps)$
of a data sample $j$ for a source $k$ is defined as the integral over the energy
of the product of the detector effective area and the differential flux
$\frac{\mathrm{d}\Phi}{\mathrm{d}E}$ of the source:
\begin{equation}
 \mathcal{Y}_{\mathrm{s},j}(\xs,\ps) \equiv \int_0^\infty \mathrm{d}E A_{\mathrm{eff},j}(E)|_{\xsk} \frac{\mathrm{d}\Phi}{\mathrm{d}E}(E,\ps) T_{\mathrm{live},j}
\label{eq:Ysj}
\end{equation}
It is the mean number of signal events expected from a source with source
parameters $\ps$ at position $\xs$. In the most-general case, the source position $\xs$
consists of three quantities: right-ascention, declination, and observation time,
i.e. $\xs = (\alpha_{\mathrm{s}},\delta_{\mathrm{s}},t_{\mathrm{obs}})$.

\subsection{Effective Area}

In Skylab the effective area $A_{\mathrm{eff},j}$ of a data sample $j$ is not
calculated separately in order to avoid binning effects. However, the effective
area can be calculated using the monte-carlo weights \code{mcweight}\footnote{In IceCube
known as ``OneWeight'', but which already includes the number of used MC files.}
of the simulation events.
The monte-carlo weights have the unit GeV~cm$^2$~sr.
Using the monte-carlo weight, $w_{m,j}$, of the $m$th event of data sample $j$
the effective area is given by the sum over the event weights divided by the
solid angle and the energy range $\Delta E$ of the summed selected events:
\begin{equation}
 A_{\mathrm{eff},j}(E) = \frac{\sum_{m=1}^{M} w_{m,j}}{\Omega \Delta E}
\end{equation}


\subsection{The DetSigEff Class}

\class{DetSigEff} provides a detector signal efficiency class to
compute the integral given in equation \eq{eq:Ysj}. The detector signal
efficiency depends on the flux model and its source parameters, which might
change during the likelihood maximization process. It is also dependent on the
detector effective area, hence detector dependent. Thus, \class{DetSigEff} must
be provided with a detector signal efficiency implementation method derived from
the \class{DetSigEffImplMethod} class.

\subsubsection{The DetSigEffImplMethod Class}

\class{DetSigEffImplMethod} defines the interface between the detector signal
efficiency implementation method and the \class{DetSigEff} class.

% List of detector signal efficiency implementation methods.
Table \ref{tbl:I3DetSigEffImplMethod} lists all available IceCube specific
detector signal efficiency implementation methods and their description.
\begin{table}
\caption{IceCube specific detector signal efficiency implementation methods.}
\label{tbl:I3DetSigEffImplMethod}

\begin{tabular}{l | p{10cm}}
\hline
I3FixedFluxDetSigEff & Detector signal efficiency implementation method for a
    fixed flux model, which might contain flux parameters, but which
    are not fit in the likelihood maximization process.
    This implementation assumes that the detector effective
    area depends solely on the declination of the source. This method creates
    a spline function of given order for the logarithmic values of the
    $\sin(\delta)$-dependent detector signal efficiency.

    The constructor of this implementation method requires a $\sin(\delta)$
    binning definition for the monte-carlo events and the order of the spline
    function.
\end{tabular}
\end{table}



\section{Implemented Likelihood Models}
This section describes the implemented likelihood models.

\subsection{Time Dependent Point-Source Flare}

The \class{TimeDepPSFlareLHModel} class provides the likelihood model for searching for a point source with unknown time-dependence.
The search is based on the formulism described in \cite{TimeDepPSSearchMethods2010}.

The model utilizes a two-component likelihood function with signal and background events.

\bibliographystyle{unsrt}
\bibliography{biblio}

\end{document}
